\documentclass[12pt]{article}
\usepackage[utf8]{inputenc}
\usepackage[left=2.5cm, right=2.5cm]{geometry}
\usepackage{svg}
\usepackage{eurosym}

\title{
\vspace{-3cm}
\begin{figure}[h]
  \centering
  \includesvg[width=10em]{logo.svg}
\end{figure}
\vspace{-0.5cm}
DeltaX \\
\vspace{0.3cm}
Candidatura per il progetto di Ingegneria del Software
}
\author{
Dardouri Leila, Lazari Alberto, Micheletti Christian, \\
Pavan Riccardo, Stafa Diego, Stevanato Giacomo, Trentin Daniele \vspace{0.4cm}\\
deltax.swe@gmail.com
}
\date{17 Novembre 2021}

\begin{document}

\maketitle

\section*{Candidatura e promessa di consegna}

Il gruppo \textit{DeltaX} si candida per il capitolato C3 \textit{"CC4D"} di SanMarco Informatica S.p.A., impegnandosi a consegnare il prodotto richiesto dal proponente entro l'11 Aprile 2022. 

\section*{Preventivo costi e impegno orario}

Stimiamo un preventivo di 14.980 \euro \ per 700 ore di lavoro totali, distribuite per ruolo e tra i vari componenti del gruppo nei seguenti modi:

\begin{table}[h]
    \centering
    \begin{tabular}{|c|c|}
        \hline
        \textbf{Membro} & \textbf{Impegno orario} \\
        \hline
        Dardouri Leila       & 100 \\
        Lazari Alberto       & 100 \\
        Micheletti Christian & 100 \\
        Pavan Riccardo       & 100 \\
        Stafa Diego          & 100 \\
        Stevanato Giacomo    & 100 \\
        Trentin Daniele      & 100 \\
        \hline
        \textbf{Totale} & 700 \\
        \hline
    \end{tabular}
    \quad\quad\quad\quad\quad
    \begin{tabular}{|c|c|c|}
        \hline
        \textbf{Ruolo} & \textbf{Ore} & \textbf{Costo} \\
        \hline
        Responsabile    & 119 & 3.570 \euro \\
        Amministratore  &  63 & 1.260 \euro \\
        Analista        & 119 & 2.975 \euro \\
        Progettista     & 119 & 2.975 \euro \\
        Programmatore   & 175 & 2.625 \euro \\
        Verificatore    & 105 & 1.575 \euro \\
        \hline
        \textbf{Totale} & 700 & 14.980 \euro \\
        \hline
    \end{tabular}
\end{table}

\section*{Motivazione della scelta}

Abbiamo scelto questo capitolato perché:
\begin{itemize}
    \item è utile: vediamo concrete applicazioni in vari ambiti di produzione e non solo;
    \item tocca diverse aree dello sviluppo software, ad esempio sia frontend che backend;
    \item si presta a una soluzione modulare ed elegante;
    \item ha dei requisiti chiari, che non lasciano troppo spazio all'interpretazione;
    \item ci lascia la libertà di scegliere le tecnologie che preferiamo;
    \item abbiamo avuto una buona impressione dell'azienda, sia durante la presentazione che durante l'incontro.
\end{itemize}

\section*{Resoconto incontro con il proponente}

Abbiamo effettuato un incontro con il proponente in data 5 Novembre, la discussione ha chiarito e approfondito alcuni aspetti: l'accesso come admin sarà possibile solo dalla rete aziendale, mentre per il resto degli utenti, che possono solo visualizzare i grafici, anche dall'esterno.
Per la rilevazione dei dati è richiesta una API token permanente, configurabile in un relativo file. Le rilevazioni anomale dovranno essere marcate diversamente nei grafici. L'utente potrà visualizzare tutte le macchine inserite e per ognuna lo storico completo delle rilevazioni.
Il proponente preferisce che vengano utilizzati i container Docker.
Il testing non sarà attuabile su macchine reali, ma su dataset forniti con dati verosimili.
L'applicazione deve supportare un carico dati non inferiore a una rilevazione al secondo per caratteristica ed essere in grado di gestirne più di una in parallelo.
Per quanto riguarda le carte di controllo, verranno forniti dei link e dei manuali sui quali documentarsi a riguardo. 
Ci è stata infine garantita disponibilità a effettuare chiamate durante tutto il progetto per ottenere feedback e fare domande.

\end{document}
